\documentclass{article}
\usepackage{amsfonts, amsbsy, amssymb, amsmath, graphicx, float}
\usepackage{hyperref}
\hypersetup{
    colorlinks=true,
    linkcolor=blue,
    filecolor=magenta,      
    urlcolor=cyan,
}
\usepackage{subfigure}
\usepackage[numbers,sort&compress]{natbib}
%%%%%%%%%%%%%%%%%%%%%%%%%%%%%%%%%%%%%%%%%%%%%%%%%%%%%



\title{cirque project champs}
\author{BAS, VGG, FGM}
\date{May 2020}

\begin{document}

\maketitle

\section{Introduction}


Single VdW potential \cite{Soley2018}. It models the long-range interaction describing dipole-dipole attraction between neutral molecules/atoms

\begin{equation}
    V(r) = \frac{- C_6}{(\beta r^2 + \alpha)^3}
    \label{eq:vdw_single}
\end{equation}

$C_6$ is the van der Waals dispersion coefficient, and $\alpha$ and $\beta$ are parametrisation constants.

The potential's minimum is located at position $r = 0$, with energy 

\begin{equation*}
    V\left( r = 0\right) = - \frac{C_6}{\alpha^3}
\end{equation*}

If the potential is defined in a 2D plane in terms of the polar coordinates $(r, \theta)$, that is, $V = V(r, \theta)$ then, the potential well is symmetric around $r = 0$ for any $ -\pi \leq \theta \leq \pi$

\section{Double VdW Potential}

To construct a double-well potential using the formula \eqref{eq:vdw_single}, we need to vary the separation between the minima of two overlapped potentials




\newpage
\bibliographystyle{siam}
\bibliography{cirque}

\end{document}
